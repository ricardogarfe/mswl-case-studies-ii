\documentclass[11pt]{scrartcl}
\usepackage[parfill]{parskip}
\usepackage{graphicx}
\usepackage{booktabs}
\usepackage{tabulary}
\usepackage{float}
\usepackage{eurosym}
\usepackage{hyperref}

\graphicspath{{images/}}

\title{\textbf{WebKit}}
\subtitle{Subtitle}
\author{Ricardo Garc\'ia Fern\'andez}
\date{\today}

\begin{document}

\maketitle

\vfill

\begin{flushright}
    \copyright  2013 Ricardo Garc\'ia Fern\'andez - ricardogarfe [at] gmail [dot] com.

    This work is licensed under a Creative Commons 3.0 Unported License.
    To view a copy of this license visit:
 
    \url{http://creativecommons.org/licenses/by/3.0/legalcode}.
\end{flushright}

\begin{figure}[h]
    \begin{flushright}	
        \includegraphics{by}
        \label{fig:by}
    \end{flushright}
\end{figure}

\newpage

\section{WebKit}

Que es WebKit ?
Inicios de 2005 Apple era el \'unico contribuidor de WebKit.

Objetivos principales

No es un navegador Web, es un renderizador de html. A diferencia de Gecko que es un motor de rendering web para Firefox.

No estaba pensado para ser adapatado a distintas plataformas.

Curioso que sea opensources.

Licencias BSD. Apple Safary, Google con Chromium.
GPL para enlazar con código privativo.
Compatibility and Standards.
Stability.
Seguridad.
Portabilidad. Genérico para cualquier plataforma. Se ha basado en interfaces para poder implementar cada módulo específico de una plataforma.
Usabilidad. No tiene interfaz de usuario.
Hackability. 2.000.000 de líneas de código. Ha de estar bien estructurado.

\section{Where is WebKit?}
\label{sec:where-is}

Nintendo 3Ds, Kindle, Android, etc\ldots

% section where-is (end)

\section{Architecture}
\label{sec:architecture}

Application - API

WebKit - Render

WebCore - Interfaces

platform - JavaScriptCore

Abril 2010 - WebKit 2.0

% section architecture (end)

\section{WebKit 2.0}
\label{sec:webkit-2.0}

Se ha divido la API en dos procesos diferentes: UI Process y Web Process.
Como el modelo Chromium para gestionar las pestañas por procesos.

Tres modelos de procesos:
\begin{itemize}
	\item Un proceso para todo.
	\item Un proceso cada web.
	\item Un proceso conjunto de páginas web (misma url)
\end{itemize}

Estabilidad y Seguridad.

Diferencias entre WebKit2 y Chromium:

El límite de WebKit2 se encuentra en el API en vez de en la aplicación Chromium que es la encargada de separar los procesos.

% section webkit-2.0 (end)

\section{Main Ports}
\label{sec:main-ports}

\begin{itemize}
	\item Apple Mac
	\item Apple Windows
	\item Chromium
	\item GTK+
	\item Qt
	\item Cairo-based Windows
	\item EFL
	\item Windows CE
	\item BlackBerry
\end{itemize}

GTK+ Port

API GTK+ Application.
WebKitGTK+
WebCore
suop, cairo.. JavaScriptCore

% section main-ports (end)

\section{Contributors}
\label{sec:contributors}

WebKit in numbers:
\begin{itemize}
	\item Contributors, more than 450 (68 not committers).
	\item 384 committers (260 not reviewers).
	\item 124 reviewers, (51 Apple, 41 Google, 9 Nokia, 6 RIM, Igalia 5, 3 Samsung,\ldots)
\end{itemize}

Build Fixes.
Gardening. Bot Test.

\subsection{Committers}
\label{sub:committers}

Para ser 'committer' has de ser nominado por los reviewers. Tener un número mínimo de parches significativos (ronda entre los 10 y 20). Proceso burocrático y recibir la aprobación de 3. Se ha de buscar la pluralidad poniendo en contacto entre uno a más reviewers.

Cualquier compañía con más de tres reviewers puede dar permiso a un reviewer.

% subsection committers (end)

\subsection{Reviewers}
\label{sub:reviewers}

Elegido por reviewers de distinta compañía, más de 80 parches funcionales.

% subsection reviewers (end)

\section{Contribute Code}
\label{sec:}

% section  (end)
\subsection{Early Warning Systems: EWS}
\label{sub:ews}

Sistema de validación mediante boots para cada parche que se publica a través de bugzilla\url{http://webkit-commit-queue.appspot.com/}.

Se ejecutan las pruebas de compilación en cada una de los ports y se ejecutan los tests en todas las plataformas. Si existe algún error informa del mismo al usuario que ha añadido el parche.

% section ews (end)

\subsection{Landing the patch}
\label{sub:patch}

Mediante webkit-patch-land se automatiza la publicación de un parche
Changelog.
Commit.
Bugzilla.
Cierra el bug.

% subsection patch (end)

\subsection{Whatching the bots}
\label{sub:wtbots}

Responsibility for a patch does not end with the patch landing in the tree.

% subsection wtbots (end)
% section contributors (end)
\section{Technology}
\label{sec:technology}

Unificación de una solución.

% section technology (end)


\end{document}
