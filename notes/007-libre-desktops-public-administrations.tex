\documentclass[11pt]{scrartcl}
\usepackage[parfill]{parskip}
\usepackage{graphicx}
\usepackage{booktabs}
\usepackage{tabulary}
\usepackage{float}
\usepackage{eurosym}
\usepackage{hyperref}

\graphicspath{{images/}}

\title{\textbf{Libre Desktops and Public Administrations}}
\subtitle{Jos\'e Gato Luis}
\author{Ricardo Garc\'ia Fern\'andez}
\date{\today}

\begin{document}

\maketitle

\vfill

\begin{flushright}
    \copyright  2013 Ricardo Garc\'ia Fern\'andez - ricardogarfe [at] gmail [dot] com.

    This work is licensed under a Creative Commons 3.0 Unported License.
    To view a copy of this license visit:
 
    \url{http://creativecommons.org/licenses/by/3.0/legalcode}.
\end{flushright}

\begin{figure}[h]
    \begin{flushright}	
        \includegraphics{by}
        \label{fig:by}
    \end{flushright}
\end{figure}

\newpage

\section{Introduction}
\label{sec:introduction}

Jos\'e Gato Luis @jgatoluis

\par Medir el \'exito de la implantaci\'on de un escritorio basado en software libre (FLOSS).

% section introduction (end)

\section{N\'umeros}
\label{sec:numeros}

Gasto p\'ublico en licencias, la sangr\'ia de microsoft.

\par Como ?
\begin{itemize}
	\item Equilibrar el gasto, reinversi\'on en conocimiento para una comunidad aut\'onoma.
	\item Posibilidad de desarrollar soluciones propias.
\end{itemize}

\subsection{Casos de \'Exito}
\label{sub:success}

Distribuciones de Linux en Andaluc\'ia y Extremadura.
Cenatic - Implantaci\'on de las tecnolog\'ias de la informaci\'on.

% subsection success (end)
% section numeros (end)

\section{Ministerio de Administraciones P\'ublicas}
\label{sec:map}

Miguel Angel Amutio - Jefe del \'Area de Planificaci\'on y Explotaci\'on. \emph{Propuesta de recomentdaciones a la Administraci\'on General del Estado sobre utilizaci\'on del software libre y de fuentes abiertas}\footnote{\url{http://administracionelectronica.gob.es/?_nfpb=true&_pageLabel=P803324061272301226576&langPae=es&detalleLista=PAE_000001307}}.

\begin{itemize}
	\item Estudio global de qu\'e es el Software Libre para definirlo, explicarlo y ponerlo en funcionamiento en las Administraciones P\'ublicas.
	\item Representaci\'on del Software como Servicio.
	\item Dar la posibilidad de negociaci\'on a partir del Software Libre alej\'andose de la esclavitud a un Software Privativo.
	\item Proteger la inversi\'on frente a un \emph{Single vendor} que puede desaparecer.
	\item Interoperabilidad.
\end{itemize}

\par Responsabilidad de gesti\'on del software a partir de las fuentes abiertas.

% section map (end)

\section{OpenSource Software}
\label{sec:floss-2008}

\begin{itemize}
	\item Cenatic: Libresoft + Telef\'onica I+D
	\item Quantitative analysis of the public administration.
	\item Based on the experience of government employees (interviews).
\end{itemize}



% section floss-2008 (end)
\end{document}
