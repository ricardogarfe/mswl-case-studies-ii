\documentclass[11pt]{scrartcl}
\usepackage[parfill]{parskip}
\usepackage{graphicx}
\usepackage{booktabs}
\usepackage{tabulary}
\usepackage{float}
\usepackage{eurosym}
\usepackage{hyperref}

\graphicspath{{images/}}

\title{\textbf{Mozilla y Mozilla Hispano}}
\subtitle{Guillermo L\'opez,\\twillyaranda@mozilla-hispano.org}
\author{Ricardo Garc\'ia Fern\'andez}
\date{\today}

\begin{document}

\maketitle

\vfill

\begin{flushright}
    \copyright  2013 Ricardo Garc\'ia Fern\'andez - ricardogarfe [at] gmail [dot] com.

    This work is licensed under a Creative Commons 3.0 Unported License.
    To view a copy of this license visit:
 
    \url{http://creativecommons.org/licenses/by/3.0/legalcode}.
\end{flushright}

\begin{figure}[h]
    \begin{flushright}	
        \includegraphics{by}
        \label{fig:by}
    \end{flushright}
\end{figure}

\newpage

\section{Guillermo L\'opez}

{\tiny willyaranda@mozilla-hispano.org}

\section{Historia}
\label{sec:historia}

Netscape 1998 hasta 2001 Internet Explorer.

El 31 de Marzo de 1998 Netscape libera el c\'odigo fuente. Nace Mozilla.org. Internamente se llamaba as\'i a Netscape.

% section historia (end)

\subsection{Web como mercado}
\label{sub:web-mercado}

Microsoft despliega su poder para el acceso a internet a trav\'es de IE. Ejerciendo pr\'acticas monopol\'isticas.

Reacciones encontradas debido a la donaci\'on de c\'odigo de Netscape entre el parecido abandono de Netscape y la oportunidad de utilizar esos recursos para crear un navegador est\'andar y competidor frente a los dem\'as.

IExplorer copaba pr\'acticamente la cuota del mercado con el 90\%.

\textbf{Primera Versi\'on - 0.6}: 6 de Diciembre del a\~no 2000.

Mozilla 1.0 - 4 a\~nos y dos meses despu\'es de la liberaci\'on.

% subsection web-mercado (end)

\subsection{Fundaci\'on Mozilla}
\label{sub:foundation}

\emph{15 de Julio de 2003}

Todas las personas puedan generar su propio contenido, una web participativa.

% subsection foundation (end)

\section{Productos}
\label{sec:productos}

\begin{itemize}
	\item Firefox - No portabilidad a ios por el renderizado debido a gecko. Apple no permite distribuir con otro motor.
	\item FirefoxOS - 
	\item Thunderbird - no soporta exchange por eso no tiene tanto nicho. Los desarrolladores han pasado a Mozilla.
	\item Camino - 
	\item Seamonkey - firefox, thunderbird, chat, editor web.
	\item Sundbird/Lightning - 
	\item XULRunner - Parecido a una Máquina virtual de Java, el protocolo que permite la multiplataforma.
\end{itemize}

% section productos (end)
\subsection{Productos Web}
\label{sub:web-products}

\begin{itemize}
	\item Bugzilla - Bug tracking system
	\item Socorro -
	\item LXR - 
	\item MXR - Mozilla Cross Reference
\end{itemize}

% subsection web-products (end)

\subsection{Labs}
\label{sub:labs}

\begin{itemize}
	\item Weave - Sincronizaci\'on de datos entre navegadores.
	\item Bespin - Editor de c\'odigo online a trav\'es de un Canvas porque ten\'ia m\'as rendimiento que un cuadro de texto.
	\item JetPack - Interfaz para crear extensiones.
	\item TestPilot - 
	\item RainDrop - Thunderbird, unir las redes sociales en una misma p\'agina.
	\item Prism - Soulrunner - 
	\item Persona - temas ligeros
	\item Ubiquity - Le escribias al navegador lo que quer\'ias hacer. busca un mapa de madrid a fuenlabrada.
	\item Identity - Persona BrowserID - OpenID pero con tu cuenta de email.
\end{itemize}

% subsection labs (end)

\subsection{Referencias}
\label{sub:references}

\begin{itemize}
	\item DevMo
	\item WikiMo
	\item SUMO
	\item SUMOMO
\end{itemize}

% subsection references (end)

\section{Apoyando FLOSS}
\label{sec:floss}

\begin{itemize}
	\item Cairo, SKia: gr\'aficos.
	\item SQLite
	\item GTK, QT, Hildon
	\item \ldots
\end{itemize}

% section floss (end)

\section{Estandares abiertos}
\label{sec:open-standards}

\begin{itemize}
	\item  CSSS, HTML, Javascript
	\item XUL, interfaces dentro del c\'odigo de Mozilla.
\end{itemize}

% section open-standards (end)

\subsection{Basados}
\label{sub:mozilla-based}

\begin{itemize}
	\item Komodo
	\item Songbird
	\item KompoZer
	\item Bluegriffon
	\item TomTom
\end{itemize}

% subsection mozilla-based (end)

\section{Desarrollo}
\label{sec:}

Paquetes

Owner
Peer

Cualquiera puede sugerir cambios en el c\'odigo.

Owner Mike O'Connor. Meritocracia pero liderazgo de esta persona.

Bugzilla

Parche - Module Owner - Super Review.

Trunk check-in, Release Team QA, Release Branch  - Check-in trocear commits.

Tests ? 

Buildbot

% section  (end)

\subsection{Aprobar cambios}
\label{sub:valid-commits}

No todo el mundo

Errores, Fallos, Mejoras, Sugerencias, etc\ldots

Asignar un bug, votar, Problema de los duplicados.

Comentar en cualquier bug.

% subsection valid-commits (end)

\subsection{Herramientas}
\label{sub:tools}

\begin{itemize}
	\item Bugzilla
	\item newsgroups
	\item irc
	\item Wiki
	\item 
\end{itemize}

\section{Furtherwork}
\label{sec:furtherwork}

JavaScript2
Estandar de V\'ideo \emph{Web M}. MP4 abierto hasta 2017.
Audio ogg pero pasando a mp3.
Dispositivos m\'oviles.

% section furtherwork (end)

\section{Metas}
\label{sec:goals}

Privacidad innovaci\'on, drumbeat (web hacking)

% section goals (end)

\section{Mozilla Hispano}
\label{sec:moz-hispano}

Menthoring.
Organizaci\'on.
Siempre hay alguien que sabe m\'as que t\'u.
Tu opini\'on no es nunca impepinable.

% section moz-hispano (end)
\end{document}
