\documentclass[11pt]{scrartcl}
\usepackage[parfill]{parskip}
\usepackage{graphicx}
\usepackage{booktabs}
\usepackage{tabulary}
\usepackage{float}
\usepackage{eurosym}
\usepackage{hyperref}

\graphicspath{{images/}}

\title{\textbf{Plan9 Bell Labs}}
\subtitle{Enrique}
\author{Ricardo Garc\'ia Fern\'andez}
\date{\today}

\begin{document}

\maketitle

\vfill

\begin{flushright}
    \copyright  2013 Ricardo Garc\'ia Fern\'andez - ricardogarfe [at] gmail [dot] com.

    This work is licensed under a Creative Commons 3.0 Unported License.
    To view a copy of this license visit:
 
    \url{http://creativecommons.org/licenses/by/3.0/legalcode}.
\end{flushright}

\begin{figure}[h]
    \begin{flushright}	
        \includegraphics{by}
        \label{fig:by}
    \end{flushright}
\end{figure}

\newpage

\section{Plan9}

\emph{Unix no está preparado para trabajar en red.}

\par Plan9 fue creado por los mismos creadores de Unix.

\par Problemas con la duplicación de funcionalidades al crear una distribución distribuida.

\emph{Todo es un fichero en Unix}. Leer de un fichero virtual ON u OFF para gestionar las luces.

    echo on >> /dev/room13/lig

\par Procesos como ficheros.

\par Protocolo llamado \textbf{9P} para exportar los ficheros a través de la red.

% section history (end)
\section{History}
\label{sec:history}

\begin{itemize}
	\item 1992 First edition: University pack.
	\item 1995 ...
	\item 2002
\end{itemize}

\section{License}
\label{sec:license}

Artículo de Stallman sobre Plan9\footnote{\url{http://www.gnu.org/philosophy/plan-nine.html}}.

\par Ahora es aceptada por OSI, FSF y Debian Guidelines\footnote{\url{http://plan9.bell-labs.com/plan9/license.html}}. 

\begin{itemize}
	\item Es estilo MIT y Apache.
	\item Incompatible con GPL. Ligado a las leyes americanas.
	\item Similar a IBM License a diferencia de poder distribuir el código fuente con el trabajo derivado.
\end{itemize}

% section license (end)

\section{Contributions}
\label{sec:contributions}

\par Descarga del código fuente y aportar cualquier solución.

\par Selección y evaluación del código que se sube. No hay integración continua ?

% section contributions (end)


\end{document}
