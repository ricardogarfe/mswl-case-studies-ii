
\begin{tabular}\href{http://2.bp.blogspot.com/-CpyD6oSjkNA/UVB6ml-9HRI/AAAAAAAAEKk/eo0fBjPPa2E/s1600/zaragoza.gif}{
\includegraphics{http://2.bp.blogspot.com/-CpyD6oSjkNA/UVB6ml-9HRI/AAAAAAAAEKk/eo0fBjPPa2E/s640/zaragoza.gif}} \\ 
Ayuntamiento de Zaragoza FLOSS Migration by\nolinebreakEduardo Romero
\end{tabular} How is possible a FLOSS migration in Public Administrations ? Here we have \textit{Zaragoza succes case}.
\\

\subsubsection{ Brief history}\textbf{\href{http://www.zaragoza.es/ciudad/sectores/tecnologia/swlibre/proyecto.htm}{ZO2-AZLINUX}:}\nolinebreakFLOSS migration project designed and developed from 2005 in Zaragoza. Inside description we found their slogan:
\\\textit{"Migration from proprietary software to free software on the computers of municipal employees."}
\\ Migrate private software to FLOSS in Zaragoza Public Administration gradually emphasizing the migration office tools \textit{Microsoft Office} for \textit{OpenOffice} and \textit{Microsoft Windows} operating system for.
\\ They developed a migration plan including all actors: politics, tecnics and users. The whole path to achieve this goal, that all administration could work using FLOSS solutions. Thus Public Administration developments and maintainability is not locked to any private software provided and could invest in Research and Development.
\\ This plan is included within a larger plan. Which Zaragoza tries to turn one of the major European cities with greater economic development based on new information technologies and knowledge management.
\\ Zaragoza city of knowledge:\nolinebreak\href{http://www.zaragoza.es/ciudad/conocimiento/conocimiento.htm}{http://www.zaragoza.es/ciudad/conocimiento/conocimiento.htm}

\subsubsection{ Technologies Migration Plan}
\begin{tabular}\href{http://www.zaragoza.es/contenidos/azlinux/css/logo_azlinux.png}{
\includegraphics{http://www.zaragoza.es/contenidos/azlinux/css/logo_azlinux.png}} \\ 
AZLinux Logo
\end{tabular} Technologies involved to provide same functionality that private software solutions:
\begin{itemize}
	\item 2005 - 2007: Browser Firefox, Thunderbird, Multimedia. Changes soft for positive acceptance.
	\item 2008 - 2010: Office Suites, OpenOffice. Change is harder but accepted.
	\item 2009 - \textasciitilde SO: Switching to OS \href{http://zaragozaciudad.net/azlinux/}{AZLinux}.
\end{itemize} These steps \textbf{must be accompanied} by a parallel track:
\begin{itemize}
	\item \textit{Inventory}: Hardware and Software.
	\item \textit{Communication}: Managing change. Accompany the users through the traumatic process. Talks, cd testing, information exchange process, benefits of using FLOSS
	\item \textit{Technical Training}: An important point to consider.
	\item \textit{Users Training}: 8 hours 20 hours linux xp Office to OpenOffice. \textit{Remarkable time dedicated to users training related to Office Suites higher than OS change}.
\end{itemize}

\subsubsection{ How to Contribute} How could you contribute to this process ? I think the best way to start contributing is to spread this work and its success case to\nolinebreakpeople reluctant to switch to FLOSS. This way is that we have to follow, thus the main reason is the freedom to choose that FLOSS gives you as a user and of course tha hability to spread and share knowlege to everyone.
\\ After those principles, I want to highlight these links:
\\
\begin{itemize}
	\item Activities -\nolinebreak\href{http://www.zaragoza.es/ciudad/sectores/tecnologia/swlibre/actividades_swlibre.htm}{http://www.zaragoza.es/ciudad/sectores/tecnologia/swlibre/actividades\_swlibre.htm}
	\item Open data initiative -\nolinebreak\href{http://www.zaragoza.es/ciudad/risp/}{http://www.zaragoza.es/ciudad/risp/}
	\item Promote your applications with open-data -\nolinebreak\href{https://www.zaragoza.es/ciudad/enlinea/consulta_risp.xhtm}{https://www.zaragoza.es/ciudad/enlinea/consulta\_risp.xhtm}
\end{itemize}\textit{\textbf{"Open data and FLOSS to serve the people"}}