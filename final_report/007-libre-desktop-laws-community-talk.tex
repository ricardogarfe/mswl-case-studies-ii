
\begin{tabular}\href{http://www.hoytecnologia.com/img/noticias/foto_199067.jpg}{
\includegraphics{http://www.hoytecnologia.com/img/noticias/foto_199067.jpg}} \\ 
Libre Desktops and Public Administrations Community Talk by José Gato
\end{tabular} This is a different talk from others saw in the blog. \textit{Libre Desktops and Public Administrations} talk tries to measure FLOSS \textit{(Free Libre Open Source Software)} Desktop environments success in Spain Public Administration.
\\
\\ Main goals of mixing FLOSS software in projects could be resume in:
\begin{itemize}
	\item Balance costs
	\item Reinvest in \textit{"Comunidades Autónomas"}\nolinebreak(CCAA)\textit{.}
	\item Develop own solutions.
\end{itemize} Encouraging FLOSS use in Public Administrations becomes a double success for projects and CCAA. Yes because FLOSS has to be encouraged and must use in every Public project for each country, because is paid with people's money.
\\ But is not easy as it seems, because we have the responsability to spread the benefits of using FLOSS licenses, projects, documentation and help possitively to adopt these kind of advantages.
\\ FLOSS doesn't mean GRATIS. I recommend you take a look this \href{http://mastersfwlurjc.blogspot.com.es/2012/11/errores-de-percepcion-del-software-libre.html}{disambiguation in FLOSS} projects that I made before.

\subsubsection{ Transparency} Transparency is a very important word inside FLOSS world and has to be as important in Public Administrations world. All public data belongs to everyone in the country, so starting from this assertion we see the needs of FLOSS licenses in Public Administrations to continue evolving with solutions, adapting our needs and learning from others development. It's the wheel of knowledge, allways hungry.

\subsubsection{ Ministerio de Administraciones Públicas} Miguel Angel Amutio Chief of Planning and Operations area at \textit{Ministerio de Administraciones Públicas}\nolinebreakmade a study to introduce FLOSS advantages inside CCAA project development in 2005. This translation from FLOSS to human-readable content is necesary to spread FLOSS advantages inside a new area.
\\ Main topics could be summarized in:
\begin{itemize}
	\item Overall review of what is FLOSS to define, explain and put into operation in public administration.
	\item Representation of Software as a Service.
	\item Give a chance to negotiate from FLOSS away from slavery to proprietary software.
	\item Investment protection against a Single vendor that can disappear.
	\item Interoperability.
\end{itemize} They aren't weird and extrange for a Software Developer but, for common people (not Pulp's song, the others :))\nolinebreakaway from the FLOSS world, is not so simple to know these benefits or advantages that it provides. Yes, it provide us, I speak as a citizen which my (our) taxes are invested in IT solutions\nolinebreakdevelopment for public administration.\nolinebreak\textit{
\\}\textit{"Liability management software from FLOSS"}
\\ You can see the whole article in \href{http://administracionelectronica.gob.es/?_nfpb=true&amp;_pageLabel=P803324061272301226576&amp;langPae=es&amp;detalleLista=PAE_000001307}{Administración Electrónica website}.
\\ This was a very hudge step for introduction to FLOSS in Public Administrations in Spain. During those years invest in FLOSS projects grown producing, better solutions, Linux distributions, and training,\nolinebreakdevoting more resources to the knowledge than the payment of licenses, ie encouraging Public Administrations knowledge.

\subsubsection{ Cenatic \& Libresoft} In 2008 \href{http://www.cenatic.es/sobre-cenatic}{Cenatic} and \href{http://libresoft.es/about}{Libresoft} made an analysis to measure FLOSS implantation success in Public Adminstrations to retrieve more data directly from software developers, not only 'static' data from code.
\begin{itemize}
	\item Quantitative analysis of the public administration.
	\item Based on the experience of government employees (interviews).
\end{itemize}\href{http://observatorio.cenatic.es/index.php?option=com_content&amp;view=article&amp;id=39:software-de-fuentes-abiertas-para-el-desarrollo-de-la-administracion-publica-espanola-una-vision-global-2008&amp;catid=5:administraciones-publicas&amp;Itemid=21}{Here is} the complete document. I have a post from another Cenatic anlysis in FLOSS implantation success at Spanish Companies from \href{http://mastersfwlurjc.blogspot.com.es/2013/01/el-uso-del-software-libre-en-las.html}{2011 based in Cenatic analysis}.
\\
\\ After all data retrievement IMHO \textit{(In My Humble Opinion)}, the most important thing is to spread the virtues of using FLOSS in public administrations because this advantage is like a boomerang, closes the circle of knowledge,\nolinebreakreturning knowledge to people who invest in the creation and use of a public service.
\\
\\ Must be able to explain the virtues that gives us the use of FLOSS in any field on the basis that knowledge is free.