\section{Document Foundation}
\label{sec:document-foundation}

Charles H. Schulz:

\begin{quote}
    \textit{"keeping languages and cultures in this century needs to have fully complete location software"}
\end{quote}

Oracle bought Sun in 2009 (\textit{lots of stories have started here since this date :)}). This affects to \href{http://www.openoffice.org/}{OpenOffice}development before (Sun crisis didn't provide founds to the project), after (New company, new roles, new organization) and then, Oracle semi-abandoned OpenOffice efforts: disappeared references from web pages, guidance and roadmaps. All the work that OpenOffice was doing for 10 years was near todisappear The project will stop to exist and the project members had to prepare and founded \textit{The Document Foundation} and the \textbf{\href{http://www.documentfoundation.org/foundation}{Manifesto}}

\begin{quote}
    \textit{"For the past ten years, the OpenOffice.org community has developed, supported and promoted the world's leading open-source office productivity suite. We have attracted the support of tens of thousands of individuals and corporate bodies during this period. We now call on all our supporters to follow us into the next phase of our development, as we become an independent Foundation."}
\end{quote}

\par Leaded by proposing available FLOSS projects to people. From this date The Document Foundation created LibreOffice to continue development of an FLOSS Office solution to people. This was the born of \textit{LibreOffice}. Now available 4.0version while I'm writing these lines.

\subsection{Technologies}

\par The basic rule in a FLOSS project is \textit{"show me the code"}. Thus as is in LibreOffice, get into the code and communication channels to create an easy development tools forge.

\begin{itemize}
	\item \textit{Distributed repository} - Chose git as social and cultural decision instead of other options like Mercurial or Bazaar.\url{http://cgit.freedesktop.org/libreoffice}
	\item \textit{Integration and Revision} - \href{https://gerrit.libreoffice.org/}{Gerrit} project for patches integration (not feature branch).
	\item \textit{Issues Tracking System} - Bugzilla: Social and Cultural decision - \url{https://bugs.freedesktop.org/}
	\item \textit{Wiki} - For contributors, developers, translations, desing in \url{https://wiki.documentfoundation.org/Development}
	\item \textit{Release often Release Earlier} - Fixed release dates - 1 year and half - Have a \textit{\href{https://wiki.documentfoundation.org/ReleasePlan}{Public Roadmap}}. Release often and you will find bugs, and this is good because maintain developers interest.
	\item \textit{License} - LGPL 3.0 and MPL allow downstream from Apache License.
	\item \textit{Mailing Lists}: \href{http://lists.freedesktop.org/mailman/listinfo/libreoffice}{Developers}, questions and answers, etc.
	\item \textit{IRC Channel}:\href{irc://chat.freenode.net/libreoffice-dev}{\#libreoffice-dev channel}to contact developers, users, immediatly.
	\item \textit{Forums}: This channel is the most used and active by developers and users even than the mailing lists -\url{http://es.libreofficeforum.org/}.
\end{itemize} There is a reference guide for first steps with LibreOffice development available at \url{http://www.documentfoundation.org/develop/} with all detailed references.

\subsection{How to Contribute}

\par Charles aimed us to contribute without fear to the project, trying and sending our developments (bugs, features, etc) to mailing list. He saids that all work is well received and more important the code you contribute to the project belongs to you, there are no developer agreement neither ownership transferences in LibreOffice project. \textit{"No Pyramids, self interest"}.

\par To main sections where explained:

\begin{itemize}
	\item \textit{Easy hacks} -\url{https://wiki.documentfoundation.org/Development/Easy_Hacks}. \textit{'Easy' bugs are waiting for you} in this buglist. The first door to contribute and get feedback to get in touch with this big community. These bugs are prepared for people to contribute to the project as a start up page and familiar with community process. I think is a good idea.
	\item \textit{Crazy ideas} - \url{https://wiki.documentfoundation.org/Development/Crazy_Ideas}. The title explains a lot related to this idea, and I like it:\textit{"With a product like LibreOffice, there is often a floodgate of radical ideas on directions the project might take"}. Its a queue to propose new ideas from brainstorming community process, they invite you to contribute in this section will your thoughts, not all will be discussed but where most ideas come together and mixed the best ideas should appear.
\end{itemize}

\par There is a Wiki explaining the process to get involved as LibreOffice contributor: \url{https://wiki.documentfoundation.org/Development#How_to_get_involved}.

\par As in other FLOSS projects \textit{LibreOffice} has a basic guides to contribute in development process and decision making inside the community and how to decisions are made and which way the process become more open, simple and participatory.

\begin{itemize}
	\item \textit{Meritocracy} to propose and defend technological aspects but always has to be argued, always.
	\item \textit{Vote for specific things} - Voting is used lesser but for some decisions is easier to vote, release dates, conferences, workshops, etc.
	\item \textit{Do, don't talk} - Do-cracy, sometimes is better to do that wait or argue to introduce new features, bugs, etc. Distributed Control Systems helps to generate easily new solutions to share with community.
\end{itemize}

\par Process patterns:

\begin{itemize}
	\item \textit{Simple Tools} - As easy as only focus on your goal.
	\item \textit{Simple contribution process} - You came with your patch and present it to the community. No Agreements to give code owner rights, belongs to you.
	\item \textit{Loose structure (teams) }- I could say scalable teams, with a good workflow team are mutable and easy to working with.
	\item \textit{Native language} - Teams duplicate whatever they feel is useful for their project. Free to replicate for every language by each team.
	\item \textit{Transparency} - The most important aspect, transparency, to show the work and be truthful to everyone.
\end{itemize}I suggest you to visit \href{http://standardsandfreedom.net/}{Charles} website and \href{https://wiki.documentfoundation.org/Development#Finding_a_first_task:_Easy_Hacks}{Easy Hacks} section.

% section document-foundation (end)