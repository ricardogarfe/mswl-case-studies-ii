\documentclass[11pt]{scrartcl}
\usepackage[parfill]{parskip}
\usepackage{graphicx}
\usepackage{booktabs}
\usepackage{tabulary}
\usepackage{float}
\usepackage{eurosym}
\usepackage{hyperref}

\graphicspath{{images/}}

\title{\textbf{GNU Project}}
\subtitle{Jos\'e E. Marchesi}
\author{Ricardo Garc\'ia Fern\'andez}
\date{\today}

\begin{document}

\maketitle

\vfill

\begin{flushright}
    \copyright  2013 Ricardo Garc\'ia Fern\'andez - ricardogarfe [at] gmail [dot] com.

    This work is licensed under a Creative Commons 3.0 Unported License.
    To view a copy of this license visit:
 
    \url{http://creativecommons.org/licenses/by/3.0/legalcode}.
\end{flushright}

\begin{figure}[h]
    \begin{flushright}	
        \includegraphics{by}
        \label{fig:by}
    \end{flushright}
\end{figure}

\newpage

\section{GNU Project}
\label{sec:gnu-project}

Jos\'e E. Marchesi - 1998 en GNU.

GNU RecUtils.

% section gnu-project (end)

\section{Historia}
\label{sec:historia}

\par GNU is not Unix! 1983.

\begin{itemize}
	\item The GNU Project.
	\item GNU Software - Develops GNU software.
	\item The GNU System - GNU Software + external Software.
\end{itemize}

1983 February.

\par Unix more simple than VMS. Composed by different independent utilities. Desacoplado y modular, por eso la elecci\'on para clonar el sistema.

\par GNU Project is a Global Project, hasn't legal entity. RMS created FSF to convert legal resources to GNU hackers.

\subsection{1980}
\label{sub:1980}

\par Emacs para poder seguir desarrollando el sistema operativo.

\par Brian Fox Int\'erprete de comandos

\par External Software, \TeX{} adoption.

\par Technical trend: Clone and improve Unix. Eliminan limitaciones, flexibilizan el uso.

\par Hurd.

\par sed, creado por GNU Project.

% subsection 1980 (end)

\subsection{1990}
\label{sub:1990}

\par The Linux kernel integrated in GNU. Linux instead of Hurd.

\par Ports (migraci\'on) de las herramientas GNU para su uso mediante el n\'ucleo de Linux.

\par Jobs, NextStep, Mag emulando BSD, framework aplicaciones, hasta convertirse en MacOSX. GNUStep se bas\'o en este framework para la interfaz de GNU.

\par Se abandon\'o el proyecto por el comienzo de GNOME.

\par C++ Improvements in this era.

\par 1998 - Open Source Movement. Un error por no saber manejar la entrada como atajo al 'open source'. Licencia art\'istica OSI si FSF no.

\par From Cathedral(private development) to Bazaar (open development). 

\par Forks; egcs - GCC 2.95, XEmacs no happy ending, permanent fork.

% subsection 1990 (end)

\subsection{2000}
\label{sub:2000}

\par Interpreted Languages. Perl and Python.

\par Networking: GNUnet, inteutils.

\par Security: GNUpg, GNUtls.

\par GPLv3, LGPLv3, AGPLv3.

\par GPLv2 o posterior, Multilicencias, cualquier licencia posterior se puede eligir, no hace falta ser el propietario del software.

\par Organization changes. GNU Advisory commitee.

% subsection 2000 (end)

\subsection{2010}
\label{sub:2010}

\par GNU MediaGoblin

% subsection 2010 (end)

\subsection{AGPLv3 problem}
\label{sub:agpl}

\par AGPLv3 no se recomienda por la FSF. Es pr\'acticamente imposible cumplir. 

\par iText cambia a AGPLv3. Incluye la librer\'ia y despu\'es del uso del software como servicio, denuncia.

\url{https://fosdem.org/2013/schedule/event/agpl_panel/}

% subsection  (end)

% section historia (end)
\section{GNU Programs}
\label{sec:gnu-programs}

\par Problema generacional.

\par Bug core utils, POSIX masters.

\par Bugs problem\'aticos. No es atractivo para un desarrollador ? Suscripci\'on a la lista y no se entiende nada, hay un nivel muy alto.

% section gnu-programs (end)

\subsection{Homogeneous}
\label{sub:homogeneous}

\begin{itemize}
	\item Webpage
	\item Development
	\item Manual
	\item Reporting bugs
	\item Getting help
\end{itemize}

% subsection homogeneous (end)

\section{Overall Picture}
\label{sec:overall-pic}

\begin{itemize}
	\item RMS - Nombra mantenedores. GNUism.
	\item GNU Advisory Committe (2009) - Advisory, communication, transversal vision, external contact. advisory@gnu.org. 8 personas, mantenedores en mayor\'ia y vicepresidentes de FSF Am\'erica y Europa.
	\item Maintainers - Mantenedor, desarrollar principal del programa. Seguir las GNU policies.
	\item Contributors - Developers. with or without commit access. Ched Rammie mantenedor de bash.
\end{itemize}

% section overall-pic (end)

\section{Maintainership}
\label{sec:maintainership}

\begin{itemize}
	\item Small packages. Single maintainer. Nick Clifton Bin utils.
	\item Medium packages. Single or co-maintainers - Emacs.
	\item Strategic packages. Steering committees. Representantes de empresas que se representan a ellas mismas. GLibC, GCC, GDB.
\end{itemize}

% section maintainership (end)

\section{Policies}
\label{sec:policies}

\begin{itemize}
	\item Maintainers.
	\item Documentation.
	\item 
\end{itemize}

Coding standards: 

\begin{itemize}
	\item Maintainer(s) => proposal => RMS => document.
\end{itemize}

Mantenedores\footnote{\url{http://www.gnu.org/prep/maintain/}}.

% section policies (end)

\section{Ethical Matters}
\label{sec:ethical}

Herramienta t\'ecnica con un fin pol\'itico, que desaparezca el software privativo.

\begin{itemize}
	\item Copyleft.
	\item Avoid some technology because of sw patents. gif format problems.
	\item Never recommending propietary software.
	\item Never referring to non-free documentation.
	\item Free Software vs Open Source.
	\item GNU and Linux terminology.
\end{itemize}

% section ethical (end)

\section{Legal Matters}
\label{sec:legal}

\begin{itemize}
	\item GPL and LGPL.
	\item Accepting contributions. No se transfiere el copyright holder a no ser que lo pida el mantenedor o sea un paquete estrat\'egico.
	\item Eben Moglen (abogado) asignaci\'on de cpyright mediante la firma de un documento escaneado.
	\item Legally Significant Changes. more that 10 lines after 30 years of essay-error.
	\item Disclaimers. Para que los firme 'el jefe' por problemas en EEUU.
	\item Copyright notices.
	\item No Trademark acknowledgements.
\end{itemize}

% section legal (end)

\section{Program Design}
\label{sec:progam-design}

\par Programming Language C is preferred, but not mandatory.

\par POSIX and GNU extensions.

\par C99

\par Portability: 
\begin{itemize}
	\item GNU and GNU/Linux.
	\item Supporting POSIX like system is important. BSD, MacOSX.
	\item Supporting non-POSIX like systems is desirable. Windows.
	\item gnulib.
\end{itemize}

% section progam-design (end)

\section{Program Behaviour}
\label{sec:behaviour}

\begin{itemize}
	\item Obeying Standards: don't get slaved.
	\item Formatting error messages.
	\item Standard - -  help format.
	\item Command line interfaces.
	\item Graphical interfaces.
\end{itemize}

% section behaviour (end)

\section{Build System}
\label{sec:build-system}

\emph{AutoTools} is not mandatory, only:

\begin{itemize}
	\item Configure script and a Makefile.
\end{itemize}

% section build-system (end)

\section{gnulib}
\label{sec:gnulib}

\emph{gnulib}: es una librer\'ia que se usa a nivel de fuentes. Incorporas c\'odigo fuente de la gnulib en tu programa o librer\'ia. Importa el c\'odigo y lo porta.

% section gnulib (end)

\section{mail lists}
\label{sec:mail-lists}

private mail lists:
\begin{itemize}
	\item gnu-prog@gnu.org - .
	\item gnu-prog-discuss@gnu.org - most maintainers are subscribed.
	\item proofreaders@gnu.org - Traducci\'on de manuales.
	\item platform-testers@gnu.org - Continuous Integration.
\end{itemize}

% section mail-lists (end)

\section{compilers farm}
\label{sec:compilers-farm}

Huge farm of machines for CI.
Savannah VCS.
Main Servers: fencepost.gnu.org y chapters.gnu.org. GNU-like community: \emph{rw-rw-rw-}.

\emph{Los trackers de savannah no los usa nadie}

% section compilers-farm (end)

\section{Distributing GNU Software}
\label{sec:distro}

\begin{itemize}
	\item Compressed tarballs woth source code.
	\item Points; alpha \url{ftp://alpha.gnu.org/gnu/program}, release \url{ftp://ftp.gnu.org/gnu/program}.
	\item Planet to announce releases.
\end{itemize}

% section distro (end)
\section{Section Down}
\label{sec:down}

Inform GNU Project if she steps down.
TBTested Unmaintainer

% section down (end)

\section{Hacker Meetings}
\label{sec:meetings}

GSoC: 19 Slots en 2012.

% section meetings (end)

\end{document}
