\documentclass[11pt]{scrartcl}
\usepackage[parfill]{parskip}
\usepackage{graphicx}
\usepackage{booktabs}
\usepackage{tabulary}
\usepackage{float}
\usepackage{hyperref}

\graphicspath{{images/}}

\title{\textbf{Migraci\'on a FLOSS}}
\subtitle{Eduardo Romero}
\author{Ricardo Garc\'ia Fern\'andez}
\date{\today}

\begin{document}

\maketitle

\vfill

\begin{flushright}
    \copyright  2013 Ricardo Garc\'ia Fern\'andez - ricardogarfe [at] gmail [dot] com.

    This work is licensed under a Creative Commons 3.0 Unported License.
    To view a copy of this license visit:
 
    \url{http://creativecommons.org/licenses/by/3.0/legalcode}.
\end{flushright}

\begin{figure}[h]
    \begin{flushright}	
        \includegraphics{by}
        \label{fig:by}
    \end{flushright}
\end{figure}

\newpage

\section{Introducci\'on}
\label{sec:introduction}

\par Se acord\'o el uso de FLOSS en el Ayntamiento de Zaragoza en 2005 por parte de todos los grupos pol\'iticos. Esto es un hito !

\par Orientado a OpenData y servicios a los ciudadanos.

% section introduction (end)

\section{Entorno}
\label{sec:entorno}

\par 3000 Pc estudio para un entorno \emph{Heterog\'eneo}, esta parte es muy importante.

\par Convivencia de distintas empresas para la implantaci\'on de FLOSS a distintos modelos por lo que ganan experiencia en la implantaci\'on de FLOSS en administraciones p\'ublicas.

\par Dispersi\'on de centros, m\'as de 100 centros municipales.

% section entorno (end)

\section{Implantaci\'on}
\label{sec:implantacion}

\par Migraci\'on gradual, aceptaci\'on positiva de los usuarios:
\begin{enumerate}
	\item 2005 - 2007 Navegador Firefox, Thunderbird, Multimedia. Cambios blandos para la aceptaci\'on positiva.
	\item 2008 - 2010 Ofim\'atica, OpenOffice. El cambio es m\'as duro pero aceptado.
	\item 2009 - ~ SO: Cambio al sistema operativo AZLinux.
\end{enumerate}

\par Estos pasos han de estar acompa\~nados de un seguimiento en paralelo:
\begin{itemize}
	\item Inventario: Hardware y Software.
	\item Comunicaci\'on: Gesti\'on del cambio. Acompa\~nar a los usuarios a trav\'es del proceso traum\'atico. Charlas, cd de pruebas, informaci\'on del proceso del cambio, beneficios del uso del FLOSS
	\item Formaci\'on T\'ecnicos: Un punto importante a tener en cuenta.
	\item Formaci\'on Usuarios: 8 horas xp a linux 20 horas de Office a OpenOffice.
\end{itemize}

% section implantacion (end)

\section{GNU/Linux}
\label{sec:gnu-linux}

\par Suse Linux Enterprise Edition (SLED) para la gesti\'on de ficheros con el directorio de ficheros (Active Directory) de Novell a trav\'es de Novell Client for Linux.

\par Cambio a OpenSuse debido a la falta de calidad del servicio de SLED. Se compil\'o el kernel de SLED con OpenSuse para poder seguir accediendo a Novell Directory Service. Se segu\'ia dependiendo de Novell.

\par Se tom'o la decisi\'on de utilizar NCPFS (GPL) que eliminaba la obligaci\'on de depender de Novell y as\'i la elecci\'on del sistema operativo se libera.

% section gnu-linux (end)

\section{\Éxito}
\label{sec:exito}

\begin{itemize}
	\item Buena gesti\'on.
	\item Entender Software Libre.
	\item Gesti\'on del cambio.
	\item Recursos.
	\item Apoyo Pol\'itico.
\end{itemize}

% section exito (end)
\end{document}
