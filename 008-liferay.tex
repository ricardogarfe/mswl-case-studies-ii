\documentclass[11pt]{scrartcl}
\usepackage[parfill]{parskip}
\usepackage{graphicx}
\usepackage{booktabs}
\usepackage{tabulary}
\usepackage{float}
\usepackage{hyperref}

\graphicspath{{images/}}

\title{\textbf{Liferay}}
\subtitle{Jorge Ferrer}
\author{Ricardo Garc\'ia Fern\'andez}
\date{\today}

\begin{document}

\maketitle

\vfill

\begin{flushright}
    \copyright  2013 Ricardo Garc\'ia Fern\'andez - ricardogarfe [at] gmail [dot] com.

    This work is licensed under a Creative Commons 3.0 Unported License.
    To view a copy of this license visit:
 
    \url{http://creativecommons.org/licenses/by/3.0/legalcode}.
\end{flushright}

\begin{figure}[h]
    \begin{flushright}	
        \includegraphics{by}
        \label{fig:by}
    \end{flushright}
\end{figure}

\newpage

\section{Introduction}

\par Gestor de portales que tiende a gestor de contenidos y viceversa.

\section{Historia}
\label{sec:history}

\par En el año 2000 Brian Chan comenz\'o la aventura de Liferay a partir de la petici\'on del pastor de su iglesia para desarrollarle un portal para la comunidad. Bas\'o su desarrollo inicial en Epicentric e ideas de Yahoo (cita requerida).

\par En el año 2003 propuso su producto a la empresa donde trabajaba para utilizarlo en pos de Epicentric. El precio que le ofreci\'o su empresa era rid\'iculo en comparaci\'on del tiempo que hab\'ia invertido por lo que se propuso regalarlo. ELigi\'o la licencia MIT por ser la m\'as corta y lo public\'o en SourceForge.

\par Java estaba falto de soluciones con respecto a los portales web.

\par Educamadrid\footnote{Educamadrid - \url{http://www.educa2.madrid.org/educamadrid/}} lo tom\'o como producto central para la gesti\'on de portales en vez de Oracle Portals por ser FLOSS. Aunque s\'olo cubr\'ia un 30\% de los requerimientos el mero hecho de ser FLOSS les hizo tomar la decisi\'on.

\par Brian junt\'o a varios amigos para crear la empresa cuando se vio 'asaltado' por las llamadas de empresas que quer\'ian implementar mejoras para Liferay.

\par Mientras Brian se dedicaba a continuar desarrollando por las noches Liferay por el d\'ia el modelo de negocio de consultor\'ia incrementaba.

% section history (end)

\section{Modelo de Negocio}
\label{sec:business-model}

\par \emph{"El modelo de negocio ha de ser recurrente"}. Si consigues clientes y acabas trabajos con ellos, entonces has de volver a empezar desde cero en el siguiente ciclo por eso el modelo de SaaS está tan extendido ya que se paga por obtener un servicio y se mantine a\~no a a\~no.

\par El soporte tuvo un \'exito poco reconocido, el \emph{expertise} no triunfaba demasiado.

\par Versi\'on Enterprise y Community para que las empresas tuvieran una estabilidad respecto a los bugs. De esta forma ofrec\'ian una respuesta durante 4 a\~nos as\'i las empresas ten\'ian aunmentaron la confianza en el producto por encima del soporte.

\par Crece de forma org\'anica, es decir el dinero que gana lo invierte en la empresa a diferencia de haber aceptado el dinero de una empresa de capital riesgo.

\par El cambio de modelo de negocio se produjo debido a que es FLOSS. Software para empresas no para usuario. 20000 euros anuales para dar soporte en un producto empresarial con licencias 100000 200000 euros.

\par proyecto sostenible, \'exito sostenible, empresa sostenible, desarrollo sostenible.

\par OpenSaaS, onPremise. What's next ?

% section business-model (end)

\end{document}
